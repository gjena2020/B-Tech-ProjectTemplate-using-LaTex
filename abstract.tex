\clearpage
\phantomsection
\addcontentsline{toc}{chapter}{ABSTRACT}
\chapter*{ABSTRACT}
I trained a Convolutional Neural Network (CNN) to map raw pixels from three front-facing cameras placed on the dash-board directly to steering commands~\cite{neil2012s}. This end-to-end approach proved surprisingly powerful. With minimum training data from humans the system learns to drive in traffic on local roads with or without lane markings and on highways. It also operates in areas with unclear visual guidance such as in parking lots and on unpaved roads. The system automatically learns internal representations of the necessary processing steps such as detecting useful road features with only the human steering angle as the training signal. We never explicitly trained it to detect, for example, the outline of roads. Compared to explicit decomposition of the problem, such as lane marking detection, path planning and control, our end-to-end system optimizes all processing steps simultaneously. So it eventually leads to better performance and smaller systems. Better performance will result because the internal components self-optimize to maximize overall system performance, instead of optimizing human-selected intermediate criteria, e.g., lane detection. Such criteria understandably are selected for ease of human interpretation which doesn’t automatically guarantee maximum system performance. Smaller networks are possible because the system learns to solve the problem with the minimal number of processing steps. We used an NVIDIA DevBox model and Deep Learning algorithms using Python 3.7 for training and for determining where to drive. The system operates at 30 frames per second (FPS).\\

\textit{Keywords} : Deep Learning, Regression, OpenCV, Keras, TenserFlow, NVidia model, Convolution 2D